\documentclass[]{article}
\usepackage{lmodern}
\usepackage{amssymb,amsmath}
\usepackage{ifxetex,ifluatex}
\usepackage{fixltx2e} % provides \textsubscript
\ifnum 0\ifxetex 1\fi\ifluatex 1\fi=0 % if pdftex
  \usepackage[T1]{fontenc}
  \usepackage[utf8]{inputenc}
\else % if luatex or xelatex
  \ifxetex
    \usepackage{mathspec}
  \else
    \usepackage{fontspec}
  \fi
  \defaultfontfeatures{Ligatures=TeX,Scale=MatchLowercase}
\fi
% use upquote if available, for straight quotes in verbatim environments
\IfFileExists{upquote.sty}{\usepackage{upquote}}{}
% use microtype if available
\IfFileExists{microtype.sty}{%
\usepackage{microtype}
\UseMicrotypeSet[protrusion]{basicmath} % disable protrusion for tt fonts
}{}
\usepackage[margin=1in]{geometry}
\usepackage{hyperref}
\hypersetup{unicode=true,
            pdfborder={0 0 0},
            breaklinks=true}
\urlstyle{same}  % don't use monospace font for urls
\usepackage{graphicx,grffile}
\makeatletter
\def\maxwidth{\ifdim\Gin@nat@width>\linewidth\linewidth\else\Gin@nat@width\fi}
\def\maxheight{\ifdim\Gin@nat@height>\textheight\textheight\else\Gin@nat@height\fi}
\makeatother
% Scale images if necessary, so that they will not overflow the page
% margins by default, and it is still possible to overwrite the defaults
% using explicit options in \includegraphics[width, height, ...]{}
\setkeys{Gin}{width=\maxwidth,height=\maxheight,keepaspectratio}
\IfFileExists{parskip.sty}{%
\usepackage{parskip}
}{% else
\setlength{\parindent}{0pt}
\setlength{\parskip}{6pt plus 2pt minus 1pt}
}
\setlength{\emergencystretch}{3em}  % prevent overfull lines
\providecommand{\tightlist}{%
  \setlength{\itemsep}{0pt}\setlength{\parskip}{0pt}}
\setcounter{secnumdepth}{0}
% Redefines (sub)paragraphs to behave more like sections
\ifx\paragraph\undefined\else
\let\oldparagraph\paragraph
\renewcommand{\paragraph}[1]{\oldparagraph{#1}\mbox{}}
\fi
\ifx\subparagraph\undefined\else
\let\oldsubparagraph\subparagraph
\renewcommand{\subparagraph}[1]{\oldsubparagraph{#1}\mbox{}}
\fi

%%% Use protect on footnotes to avoid problems with footnotes in titles
\let\rmarkdownfootnote\footnote%
\def\footnote{\protect\rmarkdownfootnote}

%%% Change title format to be more compact
\usepackage{titling}

% Create subtitle command for use in maketitle
\providecommand{\subtitle}[1]{
  \posttitle{
    \begin{center}\large#1\end{center}
    }
}

\setlength{\droptitle}{-2em}

  \title{}
    \pretitle{\vspace{\droptitle}}
  \posttitle{}
    \author{}
    \preauthor{}\postauthor{}
    \date{}
    \predate{}\postdate{}
  

\begin{document}

\pagenumbering{gobble}

\includegraphics{Sample_Workflow_Hourly_files/figure-latex/unnamed-chunk-2-1.pdf}

\begin{verbatim}
##  --- Base Sewer Flow (kGPD) ----------------- 
##  Weekday:  623.8 
##  Weekend:  651.5 
##  
##  --- Peaking Factor ------------------------- 
##               min     95%     99% 
##  Weekday:    0.12    2.05    2.32 
##  Weekend:    0.08    1.94    2.09 
##  
##  --- Ground Water Infiltration (kGPD) ------- 
##   5%:  271.8 ( 188.8 gpm ) 
##  95%:  614.7 ( 426.9 gpm ) 
##  99%:  721.2 ( 500.8 gpm ) 
##  
##  --- Rainfall Derived Inflow (GPM) ---------- 
##        6-hour SCS Type-II Storm 
##      MA:  103.3 
##    5-YR:  126.9 
##   25-YR:  195.8 
##  100-YR:  274.5 
##  
##  Total Volume: 9.21 kGal/inch ( 18.99 acre ) 
##  
##  --- Peak Hourly Flow (GPM) ----------------- 
##  DWF (95%):  1,315.2 
##  WWF (99%):  1,699.9 (25-YR 6-HR)
\end{verbatim}

\begin{verbatim}
## # A tibble: 2 x 4
##   isWkDay  mf10    mf5    mf1
##   <chr>   <dbl>  <dbl>  <dbl>
## 1 Weekday 0.136 0.118  0.0618
## 2 Weekend 0.132 0.0801 0.0148
\end{verbatim}


\end{document}
